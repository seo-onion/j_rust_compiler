\documentclass[12pt,a4paper]{article}
\usepackage[utf8]{inputenc}
\usepackage[spanish]{babel}
\usepackage{amsmath}
\usepackage{amsfonts}
\usepackage{amssymb}
\usepackage{graphicx}
\usepackage{listings}
\usepackage{xcolor}
\usepackage{hyperref}
\usepackage{geometry}
\usepackage{fancyhdr}
\usepackage{apacite}

\geometry{left=2.5cm,right=2.5cm,top=3cm,bottom=3cm}

% Configuración de listings para código
\lstset{
    basicstyle=\ttfamily\small,
    keywordstyle=\color{blue}\bfseries,
    commentstyle=\color{gray}\itshape,
    stringstyle=\color{red},
    numbers=left,
    numberstyle=\tiny\color{gray},
    stepnumber=1,
    numbersep=5pt,
    backgroundcolor=\color{white},
    showspaces=false,
    showstringspaces=false,
    showtabs=false,
    frame=single,
    tabsize=2,
    captionpos=b,
    breaklines=true,
    breakatwhitespace=false,
    escapeinside={\%*}{*)},
    language=C++
}

% Encabezado y pie de página
\pagestyle{fancy}
\fancyhf{}
\rhead{Compilador para Rust}
\lhead{Lenguajes de Programación}
\rfoot{\thepage}

\title{\textbf{Compilador para Rust}\\
\large Implementación en C++ con Generación de Código x86}
\author{Sebastian Hernandez Miñano\\
\small Universidad Nacional de Ingeniería}
\date{\today}

\begin{document}

\maketitle
\newpage

\tableofcontents
\newpage

\section{Introducción}

Este documento presenta la implementación de un compilador básico para el lenguaje Rust, desarrollado en C++. El compilador implementa las fases fundamentales de compilación: análisis léxico (scanner), análisis sintáctico (parser), construcción del árbol de sintaxis abstracta (AST), verificación de tipos (type checker) y generación de código ensamblador x86-64.

El proyecto incluye características avanzadas como manejo de arrays dinámicos, implementación de funciones genéricas mediante templates, y una infraestructura web completa con backend FastAPI y frontend React para facilitar la compilación y ejecución de programas Rust de manera interactiva. Adicionalmente, el sistema ha sido preparado para deployment en servicios de nube como Render y Vercel.

\section{Gramática}

La gramática del lenguaje implementado sigue una estructura BNF (Backus-Naur Form) adaptada para un subconjunto de Rust. A continuación se presenta la gramática formal del compilador:

\subsection{Gramática Formal}

\begin{lstlisting}[language=,caption={Gramática BNF del compilador}]
Program ::= (VarDec | FunDec)*

VarDec ::= 'let' ID (':' Type)? ('=' Exp)? ';'

FunDec ::= 'fn' ID ('<' TypeParams '>')? '(' ParamList? ')'
           ('->' Type)? Body

TypeParams ::= ID (',' ID)*

ParamList ::= Param (',' Param)*
Param ::= ID ':' Type

Body ::= '{' VarDec* Stm* '}'

Stm ::= AssignStm | PrintStm | IfStm | WhileStm
      | ReturnStm | FcallStm | PushStm | AssignPStm

AssignStm ::= ID '=' Exp ';'
AssignPStm ::= AccesExp '=' Exp ';'
PrintStm ::= 'print!' '(' Exp ')' ';'
IfStm ::= 'if' Exp Body ('else' Body)?
WhileStm ::= 'while' Exp Body
ReturnStm ::= 'return' Exp ';'
FcallStm ::= ID '(' ExpList? ')' ';'
PushStm ::= Exp '.push' '(' Exp ')' ';'

Exp ::= BinaryExp | UnaryExp | PrimaryExp

BinaryExp ::= Exp BinOp Exp
BinOp ::= '+' | '-' | '*' | '/' | '<' | '>'
        | '<=' | '>=' | '&&' | '||'

PrimaryExp ::= NumberExp | BoolExp | StrExp | IdExp
             | ArrExp | AccesExp | LenExp | FcallExp

NumberExp ::= DIGIT+
BoolExp ::= 'true' | 'false'
StrExp ::= '"' CHAR* '"'
IdExp ::= ID
ArrExp ::= '[' ExpList? ']'
AccesExp ::= ID '[' Exp ']' ('[' Exp ']')*
LenExp ::= Exp '.len()'
FcallExp ::= ID ('<' TypeArgs '>')? '(' ExpList? ')'

Type ::= 'i64' | 'bool' | 'str' | 'Vec<' Type '>' | ID

ExpList ::= Exp (',' Exp)*
TypeArgs ::= Type (',' Type)*
\end{lstlisting}

\subsection{Características Sintácticas}

La gramática implementada soporta:

\begin{itemize}
    \item Declaración de variables con inferencia de tipos
    \item Funciones con parámetros genéricos (templates)
    \item Estructuras de control: \texttt{if-else}, \texttt{while}
    \item Operadores aritméticos, relacionales y lógicos
    \item Arrays dinámicos con tipo \texttt{Vec<T>}
    \item Acceso indexado multidimensional a arrays
    \item Funciones de utilidad: \texttt{print!}, \texttt{push}, \texttt{len}
\end{itemize}

\section{Parser}

El parser implementa un analizador sintáctico descendente recursivo (recursive descent parser) que construye el AST a partir del flujo de tokens generado por el scanner.

\subsection{Estructura del Parser}

\begin{lstlisting}[caption={Estructura básica del Parser}]
class Parser {
private:
    Scanner* scanner;
    Token* current;
    Token* previous;

    bool match(Token::Type ttype);
    bool check(Token::Type ttype);
    bool advance();
    bool isAtEnd();

public:
    Parser(Scanner* scanner);
    Program* parseProgram();

    // Métodos de parseo
    VarDec* parseVarDec();
    FunDec* parseFunDec();
    Body* parseBody();
    Stm* parseStm();
    Exp* parseExp();
    Exp* parseE();    // Suma/resta
    Exp* parseBE();   // Multiplicación/división
    Exp* parseF();    // Factor (primarios)
};
\end{lstlisting}

\subsection{Precedencia de Operadores}

El parser implementa la precedencia de operadores mediante el método de análisis por descenso recursivo, donde cada nivel de precedencia corresponde a un método diferente:

\begin{enumerate}
    \item \textbf{parseExp()}: Operadores lógicos (\texttt{\&\&}, \texttt{||})
    \item \textbf{parseCompare()}: Operadores relacionales (\texttt{<}, \texttt{>}, \texttt{<=}, \texttt{>=})
    \item \textbf{parseE()}: Suma y resta (\texttt{+}, \texttt{-})
    \item \textbf{parseBE()}: Multiplicación y división (\texttt{*}, \texttt{/})
    \item \textbf{parseF()}: Expresiones primarias y llamadas a funciones
\end{enumerate}

\subsection{Manejo de Errores}

El parser implementa recuperación de errores mediante excepciones:

\begin{lstlisting}[caption={Manejo de errores en el Parser}]
Token* Parser::match(Token::Type ttype) {
    if (check(ttype)) {
        return advance();
    }
    throw runtime_error("Error en parse: se esperaba " +
                       Token::typeToString(ttype));
}
\end{lstlisting}

\section{AST (Árbol de Sintaxis Abstracta)}

El AST representa la estructura jerárquica del programa fuente. Cada nodo del árbol corresponde a una construcción sintáctica del lenguaje.

\subsection{Jerarquía de Clases}

\begin{lstlisting}[caption={Jerarquía de clases del AST}]
// Clase base para expresiones
class Exp {
public:
    virtual string type() = 0;
    virtual string accept(Visitor* visitor) = 0;
    virtual ~Exp() = 0;
};

// Clase base para sentencias
class Stm {
public:
    virtual string accept(Visitor* visitor) = 0;
    virtual ~Stm() = 0;
};

// Programa completo
class Program {
public:
    list<VarDec*> vdlist;
    list<FunDec*> fdlist;
    string accept(Visitor* visitor);
};

// Declaración de función
class FunDec {
public:
    string nombre;
    string tipo;
    vector<string> Ptipos;
    vector<string> Pnombres;
    vector<string> typeParams;  // Para templates
    Body* cuerpo;
    bool isGeneric() { return !typeParams.empty(); }
};
\end{lstlisting}

\subsection{Nodos de Expresiones}

El AST implementa diferentes tipos de nodos para expresiones:

\begin{itemize}
    \item \textbf{BinaryExp}: Expresiones binarias (\texttt{a + b}, \texttt{x < y})
    \item \textbf{NumberExp}: Literales numéricos (\texttt{42}, \texttt{-10})
    \item \textbf{BoolExp}: Literales booleanos (\texttt{true}, \texttt{false})
    \item \textbf{StrExp}: Literales de cadena (\texttt{"hello"})
    \item \textbf{IdExp}: Identificadores de variables
    \item \textbf{ArrExp}: Literales de array (\texttt{[1, 2, 3]})
    \item \textbf{AccesExp}: Acceso a elementos de array (\texttt{arr[i][j]})
    \item \textbf{LenExp}: Longitud de array (\texttt{arr.len()})
    \item \textbf{FcallExp}: Llamadas a funciones
\end{itemize}

\subsection{Patrón Visitor}

El AST implementa el patrón Visitor para separar las operaciones de la estructura del árbol:

\begin{lstlisting}[caption={Implementación del patrón Visitor}]
class Visitor {
public:
    virtual string visit(BinaryExp* exp) = 0;
    virtual string visit(NumberExp* exp) = 0;
    virtual string visit(Program* p) = 0;
    virtual string visit(FunDec* fd) = 0;
    // ... más métodos visit
};

string BinaryExp::accept(Visitor* visitor) {
    return visitor->visit(this);
}
\end{lstlisting}

\section{Visitor: Typechecker}

El Typechecker implementa un visitor que recorre el AST verificando la corrección de tipos y calculando información necesaria para la generación de código.

\subsection{Funcionalidades del Typechecker}

\begin{lstlisting}[caption={Clase Typechecker}]
class Typechecker : public Visitor {
public:
    unordered_map<string, string> tipos;
    unordered_map<string, int> bytes_per_funct;
    unordered_map<string, FunDec*> funciones;
    unordered_map<string, pair<FunDec*, vector<string>>>
        instancias;

    int generar(Program* program);

    // Inferencia de tipos
    string inferirTiposExp(Exp* e);
    void inferirTiposStm(Stm* s);

    // Manejo de templates
    bool esGenerico(string tipo, vector<string>& typeParams);
    string sustituirTipo(string tipo,
                        vector<string>& typeParams,
                        vector<string>& typeArgs);
    string crearSignature(string nombre,
                         vector<string>& typeArgs);
    void visitGenericFunction(FunDec* fd,
                             vector<string>& typeArgs);
};
\end{lstlisting}

\subsection{Verificación de Tipos}

El typechecker realiza las siguientes verificaciones:

\begin{itemize}
    \item \textbf{Compatibilidad de asignaciones}: Verifica que el tipo de la expresión coincida con el tipo de la variable
    \item \textbf{Operadores binarios}: Asegura que los operandos tengan tipos compatibles
    \item \textbf{Condiciones}: Verifica que las expresiones booleanas se usen en contextos de control
    \item \textbf{Retornos de funciones}: Comprueba que el tipo de retorno coincida con la declaración
    \item \textbf{Llamadas a funciones}: Valida número y tipo de argumentos
\end{itemize}

\begin{lstlisting}[caption={Verificación de tipos en expresiones binarias}]
string Typechecker::visit(BinaryExp* exp) {
    string l = exp->left->accept(this);
    string r = exp->right->accept(this);

    switch (exp->op) {
        case PLUS_OP:
        case MINUS_OP:
        case MUL_OP:
        case DIV_OP:
            if (l != "i64" || r != "i64")
                throw runtime_error(
                    "Operacion aritmetica requiere i64");
            return "i64";

        case AND_op:
        case ORR_op:
            if (l != "bool" || r != "bool")
                throw runtime_error(
                    "Operacion logica requiere bool");
            return "bool";

        case LE_OP:
        case GR_OP:
            if (l != r)
                throw runtime_error(
                    "Comparacion entre tipos distintos");
            return "bool";
    }
}
\end{lstlisting}

\subsection{Cálculo de Memoria}

El typechecker calcula el espacio necesario en el stack para cada función:

\begin{lstlisting}[caption={Cálculo de bytes por función}]
string Typechecker::visit(FunDec* fd) {
    int parametros = 0;
    for (string t : fd->Ptipos) {
        if (t == "bool" || t == "i64")
            parametros += 8;
        else
            parametros += 16; // Arrays: puntero + tamaño
    }

    locales = 0;
    fd->cuerpo->accept(this);
    bytes_per_funct[fd->nombre] = parametros + locales;

    return fd->tipo;
}
\end{lstlisting}

\section{Visitor: GenCode}

El visitor GenCodeVisitor genera código ensamblador x86-64 a partir del AST verificado.

\subsection{Estructura del Generador}

\begin{lstlisting}[caption={Clase GenCodeVisitor}]
class GenCodeVisitor : public Visitor {
private:
    ostream& out;

public:
    Environment<int> memoria;
    unordered_map<string, bool> memoriaGlobal;
    Typechecker* tc;

    int offset = -8;
    int labelcont = 0;
    bool entornoFuncion = false;
    string nombreFuncion;

    stringstream funcionesMonomorfizadas;
    set<string> funcionesGeneradas;

    int generar(Program* program);
};
\end{lstlisting}

\subsection{Convenciones de Llamada}

El generador sigue la convención de llamada System V AMD64 ABI:

\begin{itemize}
    \item \textbf{Argumentos}: Se pasan en registros \texttt{\%rdi}, \texttt{\%rsi}, \texttt{\%rdx}, \texttt{\%rcx}, \texttt{\%r8}, \texttt{\%r9}
    \item \textbf{Retorno}: Valor de retorno en \texttt{\%rax}
    \item \textbf{Stack frame}: Se usa \texttt{\%rbp} como frame pointer
    \item \textbf{Alineación}: Stack alineado a 16 bytes
\end{itemize}

\begin{lstlisting}[caption={Generación de prólogo de función}]
string GenCodeVisitor::visit(FunDec* f) {
    out << ".globl " << f->nombre << endl;
    out << f->nombre << ":" << endl;
    out << "pushq %rbp" << endl;
    out << "movq %rsp, %rbp" << endl;
    out << "subq $" << tc->bytes_per_funct[f->nombre]
        << ", %rsp" << endl;

    // Guardar parámetros en el stack
    vector<string> argRegs = {"%rdi", "%rsi", "%rdx",
                              "%rcx", "%r8", "%r9"};
    for (int i = 0; i < f->Pnombres.size(); i++) {
        out << "movq " << argRegs[i] << ","
            << offset << "(%rbp)" << endl;
        memoria.add_var(f->Pnombres[i], offset);
        offset -= 8;
    }

    f->cuerpo->accept(this);

    out << ".end_" << f->nombre << ":" << endl;
    out << "leave" << endl;
    out << "ret" << endl;
}
\end{lstlisting}

\subsection{Generación de Expresiones}

\begin{lstlisting}[caption={Generación de código para expresiones binarias}]
string GenCodeVisitor::visit(BinaryExp* exp) {
    exp->left->accept(this);
    out << "pushq %rax" << endl;
    exp->right->accept(this);
    out << "movq %rax, %rcx" << endl;
    out << "popq %rax" << endl;

    switch (exp->op) {
        case PLUS_OP:
            out << "addq %rcx, %rax" << endl;
            break;
        case MINUS_OP:
            out << "subq %rcx, %rax" << endl;
            break;
        case MUL_OP:
            out << "imulq %rcx, %rax" << endl;
            break;
        case LE_OP:
            out << "cmpq %rcx, %rax" << endl;
            out << "movl $0, %eax" << endl;
            out << "setl %al" << endl;
            out << "movzbq %al, %rax" << endl;
            break;
    }
    return string{};
}
\end{lstlisting}

\subsection{Estructuras de Control}

\begin{lstlisting}[caption={Generación de código para if-else}]
string GenCodeVisitor::visit(IfStm* stm) {
    int label = labelcont++;

    stm->condition->accept(this);
    out << "cmpq $0, %rax" << endl;
    out << "je else_" << label << endl;

    stm->then->accept(this);
    out << "jmp endif_" << label << endl;

    out << "else_" << label << ":" << endl;
    if (stm->els)
        stm->els->accept(this);

    out << "endif_" << label << ":" << endl;
    return string{};
}
\end{lstlisting}

\subsection{Optimizaciones}

El generador implementa optimizaciones básicas:

\begin{itemize}
    \item \textbf{Evaluación constante}: Condiciones booleanas constantes se evalúan en tiempo de compilación
    \item \textbf{Eliminación de código muerto}: Ramas de \texttt{if} o \texttt{while} con condiciones constantes falsas se omiten
    \item \textbf{Etiquetas compartidas}: Uso de etiquetas duplicadas para optimizar bucles infinitos
\end{itemize}

\begin{lstlisting}[caption={Optimización de while con condición constante}]
string GenCodeVisitor::visit(WhileStm* stm) {
    if (stm->condition->type() == "BoolExp") {
        if (((BoolExp*)stm->condition)->value) {
            // Bucle infinito optimizado
            int label = labelcont++;
            out << label << ":" << endl;
            stm->b->accept(this);
            out << label << ":" << endl;  // Etiqueta duplicada
            return string{};
        }
    }
    // Generación normal con condición dinámica
    // ...
}
\end{lstlisting}

\section{Arrays Dinámicas}

La implementación de arrays dinámicos utiliza memoria heap con gestión manual mediante \texttt{malloc}.

\subsection{Representación en Memoria}

Los arrays se representan mediante dos valores de 8 bytes:

\begin{enumerate}
    \item \textbf{Puntero}: Dirección de memoria del primer elemento (\texttt{\%rax})
    \item \textbf{Tamaño}: Número de elementos en el array (\texttt{\%rcx})
\end{enumerate}

\subsection{Creación de Arrays}

\begin{lstlisting}[caption={Generación de código para literales de array}]
string GenCodeVisitor::visit(arrExp* arr) {
    int type_size = 16;  // Por defecto arrays anidados
    if (arr->tipo == "bool" || arr->tipo == "i64")
        type_size = 8;

    // Reservar memoria
    out << "movq $" << arr->elements.size() * type_size
        << ", %rdi" << endl;
    out << "call malloc@PLT" << endl;
    out << "movq %rax, %rbx" << endl;

    // Inicializar elementos
    int offset = 0;
    for (Exp* exp : arr->elements) {
        exp->accept(this);
        out << "movq %rax, " << offset << "(%rbx)" << endl;
        if (type_size == 16) {
            out << "movq %rcx, " << (offset + 8)
                << "(%rbx)" << endl;
        }
        offset += type_size;
    }

    // Retornar puntero y tamaño
    out << "movq $" << arr->elements.size()
        << ", %rcx" << endl;
    out << "movq %rbx, %rax" << endl;
}
\end{lstlisting}

\subsection{Acceso a Elementos}

El acceso a arrays soporta indexación multidimensional:

\begin{lstlisting}[caption={Acceso a elementos de array}]
string GenCodeVisitor::visit(accesExp* stm) {
    int type_size = 16;
    string tipo = stm->accept(tc);
    if (tipo == "bool" || tipo == "i64")
        type_size = 8;

    int base = memoria.lookup(stm->variable);
    out << "movq " << base << "(%rbp), %r8" << endl;

    for (Exp* e : stm->indexes) {
        e->accept(this);  // Índice en %rax
        out << "movq $" << type_size << ", %rbx" << endl;
        out << "imulq %rbx, %rax" << endl;  // offset = i * size
        out << "addq %r8, %rax" << endl;     // dirección = base + offset
        out << "movq (%rax), %r8" << endl;   // cargar elemento
    }

    out << "movq %r8, %rax" << endl;
    return string{};
}
\end{lstlisting}

\subsection{Operación Push}

La operación \texttt{push} redimensiona dinámicamente el array:

\begin{lstlisting}[caption={Implementación de push}]
string GenCodeVisitor::visit(pushStm* stm) {
    int size = 16;
    string t = stm->p->accept(tc);
    if (t == "bool" || t == "i64")
        size = 8;

    // Cargar array actual
    stm->vector->accept(this);
    out << "pushq %rax" << endl;

    // Calcular nuevo tamaño
    out << "movq %rcx, %rdi" << endl;
    out << "addq $1, %rdi" << endl;
    out << "movq %rdi, 8(%rbx)" << endl;  // Actualizar tamaño

    // Reservar nueva memoria
    out << "movq $" << size << ", %rcx" << endl;
    out << "imulq %rcx, %rdi" << endl;
    out << "call malloc@PLT" << endl;

    // Copiar elementos antiguos
    out << "popq %rax" << endl;
    // ... bucle de copia ...

    // Agregar nuevo elemento
    stm->p->accept(this);
    out << "movq %rax, (%rbx)" << endl;
}
\end{lstlisting}

\subsection{Gestión de Memoria}

La gestión de memoria se realiza de forma explícita:

\begin{itemize}
    \item \textbf{Asignación}: Mediante \texttt{malloc@PLT}
    \item \textbf{Sin liberación automática}: El sistema operativo libera la memoria al terminar el programa
    \item \textbf{Reasignación}: En operaciones como \texttt{push}, se reserva nuevo espacio y se copia el contenido
\end{itemize}

\section{Templates}

La implementación de templates (funciones genéricas) utiliza el enfoque de monomorfización en tiempo de compilación.

\subsection{Declaración de Funciones Genéricas}

\begin{lstlisting}[caption={Ejemplo de función genérica}]
fn max<T>(a: T, b: T) -> T {
    if a > b {
        return a;
    } else {
        return b;
    }
}
\end{lstlisting}

\subsection{Monomorfización}

El proceso de monomorfización genera una versión especializada de la función para cada conjunto único de tipos concretos:

\begin{lstlisting}[caption={Proceso de monomorfización en Typechecker}]
void Typechecker::visitGenericFunction(FunDec* fd,
                                      vector<string>& typeArgs) {
    string signature = crearSignature(fd->nombre, typeArgs);

    // Sustituir tipos genéricos por concretos
    for (int i = 0; i < fd->Pnombres.size(); i++) {
        string tipoConcreto = sustituirTipo(fd->Ptipos[i],
                                           fd->typeParams,
                                           typeArgs);
        tipos[fd->Pnombres[i]] = tipoConcreto;
    }

    // Verificar tipos con los parámetros concretos
    fd->cuerpo->accept(this);

    // Registrar instancia para generación de código
    bytes_per_funct[signature] = parametros + locales;
}
\end{lstlisting}

\subsection{Inferencia de Tipos}

El compilador infiere los tipos de los templates a partir de los argumentos de la llamada:

\begin{lstlisting}[caption={Inferencia de tipos en llamadas genéricas}]
string Typechecker::visit(FcallExp* fcall) {
    FunDec* f = buscarFuncion(fcall->nombre);

    if (f->isGeneric()) {
        // Inferir tipos de argumentos
        vector<string> typeArgs;
        for (auto arg : fcall->argumentos) {
            string tipoArg = inferirTiposExp(arg);
            typeArgs.push_back(tipoArg);
        }

        // Crear signature única
        string signature = crearSignature(fcall->nombre,
                                         typeArgs);

        // Registrar instancia si no existe
        if (instancias.count(signature) == 0) {
            instancias[signature] = {f, typeArgs};
            visitGenericFunction(f, typeArgs);
        }

        return sustituirTipo(f->tipo, f->typeParams, typeArgs);
    }
}
\end{lstlisting}

\subsection{Generación de Código Monomorfizado}

Cada instancia genera una función separada con un nombre único:

\begin{lstlisting}[caption={Generación de funciones monomorfizadas}]
void GenCodeVisitor::generarFuncionMonomorfizada(
    FunDec* fd,
    vector<string>& typeArgs,
    string signature) {

    // Crear nombre único: max<i64> -> max_i64
    string nombreMono = crearNombreMonomorfizado(
        fd->nombre,
        typeArgs
    );

    out << ".globl " << nombreMono << endl;
    out << nombreMono << ":" << endl;

    // Generar código con tipos sustituidos
    // ...
}

string GenCodeVisitor::crearNombreMonomorfizado(
    string nombre,
    vector<string>& typeArgs) {

    string result = nombre;
    for (auto& tipo : typeArgs) {
        result += "_" + tipo;
    }
    return result;  // Ej: "max_i64"
}
\end{lstlisting}

\subsection{Ventajas de la Monomorfización}

\begin{itemize}
    \item \textbf{Rendimiento}: Código especializado sin overhead de tipos dinámicos
    \item \textbf{Verificación en tiempo de compilación}: Errores de tipos detectados antes de ejecutar
    \item \textbf{Sin penalización en runtime}: No hay necesidad de boxing o dispatch dinámico
    \item \textbf{Optimización}: El compilador puede optimizar cada versión especializada
\end{itemize}

\subsection{Limitaciones}

\begin{itemize}
    \item \textbf{Tamaño del binario}: Cada instancia genera código adicional
    \item \textbf{Tiempo de compilación}: Más funciones que verificar y generar
    \item \textbf{Sin polimorfismo runtime}: Los tipos deben conocerse en tiempo de compilación
\end{itemize}

\section{Deploy y Producción}

La arquitectura del sistema se divide en tres componentes principales desplegados en servicios cloud gratuitos.

\subsection{Arquitectura General}

\begin{figure}[h]
\centering
\begin{verbatim}
┌─────────────┐     HTTP      ┌──────────────┐
│   Usuario   │ ←────────────→ │   Vercel     │
│  (Browser)  │                │  (Frontend)  │
└─────────────┘                └──────────────┘
                                      │
                                      │ HTTPS/REST
                                      ▼
                                ┌──────────────┐
                                │   Render     │
                                │  (Backend)   │
                                │   FastAPI    │
                                └──────────────┘
                                      │
                                      │ subprocess
                                      ▼
                                ┌──────────────┐
                                │  Compilador  │
                                │    C++       │
                                └──────────────┘
\end{verbatim}
\caption{Arquitectura de deployment}
\end{figure}

\subsection{Backend: FastAPI en Render}

\subsubsection{Tecnologías Utilizadas}

\begin{itemize}
    \item \textbf{Framework}: FastAPI 0.104.1
    \item \textbf{Servidor}: Uvicorn con workers asíncronos
    \item \textbf{Containerización}: Docker con imagen Ubuntu 22.04
    \item \textbf{Plataforma}: Render.com (Free Tier)
\end{itemize}

\subsubsection{Dockerfile}

\begin{lstlisting}[language=bash,caption={Dockerfile para backend}]
FROM ubuntu:22.04

ENV DEBIAN_FRONTEND=noninteractive

RUN apt-get update && apt-get install -y \
    python3 \
    python3-pip \
    g++ \
    make \
    graphviz \
    gcc \
    && rm -rf /var/lib/apt/lists/*

WORKDIR /app

COPY backend/requirements.txt ./backend/
RUN pip3 install --no-cache-dir -r backend/requirements.txt

COPY *.cpp *.h ./
COPY Makefile ./

RUN make

COPY backend/ ./backend/

WORKDIR /app/backend

EXPOSE 8000

CMD ["python3", "-m", "uvicorn", "main:app",
     "--host", "0.0.0.0", "--port", "8000"]
\end{lstlisting}

\subsubsection{Endpoints de la API}

\begin{enumerate}
    \item \textbf{POST /compile}
    \begin{itemize}
        \item Recibe código Rust como JSON
        \item Ejecuta el compilador C++
        \item Genera: tokens, AST (DOT y PNG), ensamblador
        \item Retorna ID de compilación único
    \end{itemize}

    \item \textbf{GET /tokens/\{id\}}
    \begin{itemize}
        \item Retorna archivo de tokens generado por el scanner
    \end{itemize}

    \item \textbf{GET /ast/\{id\}}
    \begin{itemize}
        \item Retorna imagen PNG del AST generada con Graphviz
    \end{itemize}

    \item \textbf{GET /asm/\{id\}}
    \begin{itemize}
        \item Retorna código ensamblador x86-64
    \end{itemize}

    \item \textbf{GET /run/\{id\}}
    \begin{itemize}
        \item Compila el ensamblador con GCC
        \item Ejecuta el binario resultante
        \item Retorna stdout y stderr
    \end{itemize}

    \item \textbf{GET /health}
    \begin{itemize}
        \item Health check para monitoreo
    \end{itemize}
\end{enumerate}

\subsubsection{Manejo de Etiquetas Duplicadas}

El backend incluye preprocesamiento del código ensamblador para manejar la optimización de etiquetas duplicadas:

\begin{lstlisting}[language=Python,caption={Preprocesamiento de etiquetas}]
# Leer archivo .s
with open(asm_file, 'r') as f:
    lines = f.readlines()

seen_labels = set()
processed_lines = []

for line in lines:
    stripped = line.strip()
    # Detectar etiquetas while_X: duplicadas
    if stripped.endswith(':') and stripped.startswith('while_'):
        if stripped in seen_labels:
            # Comentar etiqueta duplicada
            processed_lines.append('# ' + line)
            continue
        seen_labels.add(stripped)
    processed_lines.append(line)

# Guardar archivo procesado
with open(asm_file, 'w') as f:
    f.writelines(processed_lines)
\end{lstlisting}

\subsection{Frontend: React en Vercel}

\subsubsection{Tecnologías Utilizadas}

\begin{itemize}
    \item \textbf{Framework}: React 18.2.0
    \item \textbf{Editor}: Monaco Editor (mismo de VS Code)
    \item \textbf{HTTP Client}: Axios
    \item \textbf{Build Tool}: Create React App
    \item \textbf{Plataforma}: Vercel (Free Tier)
\end{itemize}

\subsubsection{Componentes Principales}

\begin{lstlisting}[language=JavaScript,caption={Componente principal App.js}]
function App() {
  const [code, setCode] = useState(DEFAULT_CODE);
  const [compilationId, setCompilationId] = useState(null);
  const [tokens, setTokens] = useState('');
  const [assembly, setAssembly] = useState('');
  const [astImage, setAstImage] = useState(null);
  const [output, setOutput] = useState('');

  const handleCompile = async () => {
    const response = await axios.post(
      `${API_URL}/compile`,
      { code }
    );

    const id = response.data.files.id;
    setCompilationId(id);

    // Cargar resultados
    const tokensRes = await axios.get(`${API_URL}/tokens/${id}`);
    setTokens(tokensRes.data);

    const asmRes = await axios.get(`${API_URL}/asm/${id}`);
    setAssembly(asmRes.data);

    setAstImage(`${API_URL}/ast/${id}`);
  };

  const handleRun = async () => {
    const response = await axios.get(
      `${API_URL}/run/${compilationId}`
    );
    setOutput(response.data.stdout);
  };

  // Render UI con Monaco Editor y tabs
}
\end{lstlisting}

\subsubsection{Configuración de Vercel}

\begin{lstlisting}[language=JSON,caption={vercel.json}]
{
  "rewrites": [
    { "source": "/(.*)", "destination": "/index.html" }
  ]
}
\end{lstlisting}

Variables de entorno en Vercel:
\begin{verbatim}
REACT_APP_API_URL=https://j-rust-compiler.onrender.com
\end{verbatim}

\subsection{Pipeline de Deployment}

\subsubsection{Flujo de CI/CD}

\begin{enumerate}
    \item \textbf{Desarrollo local}
    \begin{itemize}
        \item Modificación de código C++, Python o React
        \item Testing local con \texttt{make} y scripts de prueba
    \end{itemize}

    \item \textbf{Git push}
    \begin{itemize}
        \item Commit de cambios al repositorio
        \item Push a rama \texttt{master} en GitHub
    \end{itemize}

    \item \textbf{Deployment automático}
    \begin{itemize}
        \item Render detecta cambios en \texttt{master}
        \item Construye imagen Docker
        \item Ejecuta \texttt{make} para compilar C++
        \item Inicia servidor FastAPI
        \item Vercel detecta cambios en \texttt{frontend/}
        \item Ejecuta \texttt{npm run build}
        \item Despliega sitio estático
    \end{itemize}
\end{enumerate}

\subsubsection{Monitoreo y Logs}

\begin{itemize}
    \item \textbf{Render}: Dashboard con logs en tiempo real, métricas de CPU/memoria
    \item \textbf{Vercel}: Analytics de builds, error tracking, performance monitoring
    \item \textbf{Health checks}: Endpoint \texttt{/health} monitoreado cada 5 minutos
\end{itemize}

\subsection{Consideraciones de Producción}

\subsubsection{Limitaciones del Free Tier}

\begin{itemize}
    \item \textbf{Render}
    \begin{itemize}
        \item El servicio se "duerme" después de 15 minutos de inactividad
        \item Primera petición tarda ~30 segundos en responder
        \item 750 horas/mes de compute time
    \end{itemize}

    \item \textbf{Vercel}
    \begin{itemize}
        \item 100GB bandwidth/mes
        \item Builds ilimitados
        \item Funciones serverless con timeout de 10 segundos
    \end{itemize}
\end{itemize}

\subsubsection{Optimizaciones Implementadas}

\begin{enumerate}
    \item \textbf{Caching de builds}
    \begin{itemize}
        \item Docker layer caching en Render
        \item Build cache de npm en Vercel
    \end{itemize}

    \item \textbf{Gestión de archivos temporales}
    \begin{itemize}
        \item Directorio temporal por compilación
        \item ID único usando \texttt{os.urandom(8).hex()}
        \item Endpoint \texttt{/cleanup} para liberar espacio
    \end{itemize}

    \item \textbf{Timeouts configurados}
    \begin{itemize}
        \item Compilación: 30 segundos
        \item Ejecución de ensamblador: 15 segundos
        \item Ejecución de binario: 10 segundos
    \end{itemize}
\end{enumerate}

\subsubsection{Seguridad}

\begin{itemize}
    \item \textbf{CORS}: Configurado para aceptar cualquier origen en desarrollo
    \item \textbf{Validación de entrada}: Pydantic valida estructura de requests
    \item \textbf{Timeouts}: Previenen ejecución infinita de código malicioso
    \item \textbf{Sandboxing}: Procesos ejecutados en contenedor Docker aislado
    \item \textbf{HTTPS}: Comunicación encriptada mediante certificados de Render y Vercel
\end{itemize}

\section{Conclusiones}

El proyecto ha logrado implementar exitosamente un compilador funcional para un subconjunto del lenguaje Rust, demostrando la aplicación práctica de los conceptos teóricos de compiladores estudiados en el curso.

\subsection{Logros Principales}

\begin{itemize}
    \item Implementación completa de las fases de compilación: scanner, parser, AST, type checking y generación de código
    \item Soporte para características avanzadas como arrays dinámicos con gestión de memoria heap
    \item Sistema de templates mediante monomorfización en tiempo de compilación
    \item Generación de código ensamblador x86-64 optimizado
    \item Infraestructura web completa con visualización interactiva del proceso de compilación
    \item Deployment en producción con arquitectura escalable y moderna
\end{itemize}

\subsection{Trabajo Futuro}

Posibles extensiones del proyecto incluyen:

\begin{itemize}
    \item Implementación de structs y enums
    \item Sistema de ownership y borrowing de Rust
    \item Optimizaciones adicionales en el generador de código
    \item Manejo automático de memoria (garbage collection o RAII)
    \item Soporte para concurrencia y paralelismo
    \item Mejoras en mensajes de error con información de línea y columna
\end{itemize}

\newpage

\bibliographystyle{apacite}
\begin{thebibliography}{99}

\bibitem{aho2006}
Aho, A. V., Lam, M. S., Sethi, R., \& Ullman, J. D. (2006).
\textit{Compilers: Principles, techniques, and tools} (2nd ed.).
Pearson Education.

\bibitem{appel2004}
Appel, A. W., \& Palsberg, J. (2004).
\textit{Modern compiler implementation in Java} (2nd ed.).
Cambridge University Press.

\bibitem{fastapi2023}
FastAPI. (2023).
\textit{FastAPI framework, high performance, easy to learn, fast to code, ready for production}.
Recuperado de https://fastapi.tiangolo.com/

\bibitem{react2023}
React. (2023).
\textit{A JavaScript library for building user interfaces}.
Recuperado de https://react.dev/

\bibitem{rust2023}
Rust Foundation. (2023).
\textit{The Rust programming language}.
Recuperado de https://www.rust-lang.org/

\bibitem{systemv2023}
System V Application Binary Interface. (2023).
\textit{AMD64 architecture processor supplement}.
Recuperado de https://refspecs.linuxbase.org/

\bibitem{graphviz2023}
Graphviz. (2023).
\textit{Graph visualization software}.
Recuperado de https://graphviz.org/

\bibitem{render2023}
Render. (2023).
\textit{Cloud application hosting for developers}.
Recuperado de https://render.com/

\bibitem{vercel2023}
Vercel. (2023).
\textit{Develop. Preview. Ship. For the best frontend teams}.
Recuperado de https://vercel.com/

\end{thebibliography}

\end{document}
